\documentclass[a4paper,10pt]{article}
%\documentclass[a4paper,10pt]{scrartcl}

\usepackage[utf8]{inputenc}
\usepackage{listings}

\title{LOBCDER configuration instructions}
\author{S. Koulouzis}
\date{}

\pdfinfo{%
 /Title  ()
 /Author  ()
 /Creator ()
 /Producer ()
 /Subject ()
 /Keywords ()
}


\begin{document}
\maketitle
This is are some basic instruction on how to configure and deploy LOBCDER 

\section{Requrements}
mysql >= 5.1
java >= 1.7
apache-tomcat>=6.0.33


\section{Create and configure the logical file system}
LOBCDER uses MySQL to store and query the logical file system.

\subsection{Create and configure MySQL DB}
Install mysql: 
\begin{lstlisting}
sudo apt-get install mysql-server mysql-client
\end{lstlisting}


Connect to the local database server as root.
\begin{lstlisting}
$mysql -h localhost -u root -p
$mysql> create database lobcderDB2;
\end{lstlisting}


Now we create the user 'lobcder' and give him super permissions on the new database. SUPER is required for 
enabling the delete and replicate triggers\footnote{When LOBCDER is asked to delete a file it simply removes it 
from it's logical file system, and schedules a deletion of the physical files which will run at a later time 
asynchronously. The same applies for replication} (for mysql version 5.0.95). 
\begin{lstlisting}
$mysql>GRANT ALL PRIVILEGES on lobcderDB2.* to lobcder@localhost \ 
			IDENTIFIED by 'password';
$mysql>GRANT SUPER ON *.* to lobcder@localhost \
			IDENTIFIED by 'password';
\end{lstlisting}


\subsection{Initialize the logical file system and the triggers}
Edit the init.sql file to include the proper backends\footnote{Backends are the storage systems 
(cloud, grid, FTP) LOBCDER is using for holding the physical files.}. 
These will be used by LOBCDER to hold the physical data. At the moment we have tested sftp and swift so if we want 
to add a swift cloud storage we add in \texttt{storage\_site\_table}, swift://localhost:8080/path. In the 
init.sql file look for the following line and modify it: 

\begin{lstlisting}
INSERT INTO credential_table(username, password) \
	 VALUES ('backend_username', 'backend_password');
SET @credID = LAST_INSERT_ID();

INSERT INTO 
	storage_site_table(resourceURI, 
			 credentialRef, 
			 currentNum, 
			 currentSize, 
			 quotaNum, 
			 quotaSize)
			 VALUES('schema://HOST:PORT/PATH/', 
			 @credID, -1, -1, -1, -1);
\end{lstlisting}

To add more storage sites copy and paste the above lines as many times as necessary. 

\subsection{Add local accounts}\lebel{leb:local}

To enable authorization and authentication based on local accounts, i.e. account created and kept on the local 
DB, in the init.sql file look for the following line and modify it:

 \begin{lstlisting}
INSERT INTO auth_usernames_table(token, uname) \
	VALUES ('password', 'username');

SET @authUserNamesRef = LAST_INSERT_ID();
INSERT INTO auth_roles_tables(roleName, unameRef) \
	VALUES ('role1',   @authUserNamesRef),
	('role2',   @authUserNamesRef),
	('role2', @authUserNamesRef);
\end{lstlisting}

Alternatively you can run this command \textbf{after} you have run the init.sql script 
\begin{lstlisting}
$mysql -h localhost -u root -p
$mysql>INSERT INTO lobcderDB2.auth_usernames_table (token, uname) 
   VALUES ('user_password', 'username');
$mysql> use lobcderDB2 database;
$mysql> select * from auth_usernames_table;
\end{lstlisting}

Take the user \texttt{ID\_NUM} from the \texttt{auth\_usernames\_table} table and type:
\begin{lstlisting}
$mysql>INSERT INTO lobcderDB2.auth_roles_tables (role_name, uname_id)
  VALUES ('role1', ID_NUM);
\end{lstlisting}


\subsection{Create the schema}
Run the init.sql script from the command-line as 'lobcder': 
\begin{lstlisting}
mysql --user=lobcder --password=password lobcderDB2 < init.sql
\end{lstlisting}


To check if the database is created you can run:
\begin{lstlisting}
mysql -h localhost -u root -p
mysql> show databeses;
\end{lstlisting}

and you should see something like this: 

\begin{lstlisting}
+--------------------+
| Database           |
+--------------------+
| information_schema |
| lobcderDB2         |
| mysql              |
+--------------------+
3 rows in set (0.00 sec)
\end{lstlisting}


to see if all storage sites are created type:

\begin{lstlisting}
mysql> select * from storage_site_table;
\end{lstlisting}
or 
\begin{lstlisting}
mysql> select * from lobcderDB2.storage_site_table;
\end{lstlisting}

and to see the local users (see \ref{leb:local})

\begin{lstlisting}
mysql> select * from lobcderDB2.auth_usernames_table;
mysql> select * from lobcderDB2.auth_roles_tables;
\end{lstlisting}


\section{Configure LOBCDER to connect to the DB}
In the META-INF/context.xml file edit or add this line:

\begin{lstlisting}

<Resource auth="Container" driverClassName="com.mysql.jdbc.Driver"
      name="jdbc/lobcder" 
      password="password"
      type="javax.sql.DataSource" 
      url="jdbc:mysql://localhost:3306/lobcderDB2" 
      username="lobcder"
      maxActive="20"
      maxIdle="100"
      minIdle="10"
      maxWait="30000"
      validationQuery="SELECT 1"
      testOnBorrow="true"
      poolPreparedStatements="true"
      removeAbandoned="true"
      removeAbandonedTimeout="60"
      logAbandoned="true"/>
\end{lstlisting}

In the WEB-INF/web.xml file edit or add this line:
\begin{lstlisting}
  <resource-ref>
    <description>DB Connection Pooling</description>
    <res-ref-name>jdbc/lobcderDB2</res-ref-name>
    <res-type>javax.sql.DataSource</res-type>
    <res-auth>Container</res-auth>
  </resource-ref>
\end{lstlisting}


\section{Configure LOBCDER properties}\label{sec:lobProp}
The file lobcder.properties contains some properties used by lobcder. 

\begin{itemize}
 \item worker.token: A hard-coded token used by the lobcder workers. Values: It can be any alphanumeric. It has to match the property in lobcder-worker/src/main/resources/auth.properties (see \ref{sec:workers})
 \item replication.aggressive: 
 \itam get.redirect: Controls if lobcder redirect GET requests to workers. Values: true/false
 \item use.metadata.repository: Controls if lobcder will send access data to a metadata repository GET requests to workers. Values: true/false
 \item metadata.reposetory.url: The url of the metadata repository. Values: URL 
 \item default.rowlimit: Controls the number of results returned by the REST service when sending this query: http://host.com/rest/items/query?path={path} Values: integer
%  \item tokens.deletesweep.count: 
  \item mi.cert.pub.der: The public key used to validate tickets. Values: file path 
  \item mi.cert.alg: The alogrithm used for validate tickets:  Values: DAS/RSA
\end{itemize}


If you have a set of workers (see \ref{sec:workers}) you can add to the lobcder/src/main/resources/workers file the worker's URL. The file may look like this:

\begin{lstlisting}
http://hos1.com/lobcder-worker/
http://hos2.com/lobcder-worker/
.
.
http://hosN.com/lobcder-worker/
\end{lstlisting}


\section{Build and Deploy LOBCDER}
LOBCDER is tested on apache-tomcat-6.0.33 but you can deploy it on any application server such as glassfish etc. 

To build the project run:
\begin{lstlisting}
mvn install
\end{lstlisting}

To have lobcder on a url like: http://host.com/lobcder/ instead of http://host.com/lobcder-2.4/ rename the target: 
\begin{lstlisting}
mv target/lobcder-2.4 target/lobcder
\end{lstlisting}

Copy the target/lobcder folder on \$CATALINA_HOME/webapps/ and restart tomcat.




\section{Configure Build and Deploy LOBCDER worker}\label{sec:workers}
The lobcder worker is a statless light version of lobcder and it is used to distribute the load of GET requests. It can be deployed on any machine as long as the master and worker can reach eachother (both should have public IP address). 

In order for a lobcder-worker to serve GET requests it has to be able to conntact the lobcder-master. To do so you have to add the master's url and credentials in the  lobcder-worker/src/main/resources/auth.properties file. The file will look like this (for configuration in lobcder-master see \ref{sec:lobProp}):

\begin{lstlisting}
rest.url=http://host.com/lobcder/rest
rest.username=WORKER_NAME
rest.password=WORKER_PASSWORD
\end{lstlisting}

To build the project run:
\begin{lstlisting}
mvn install
\end{lstlisting}

To have lobcder on a url like: http://host.com/lobcder-worker instead of http://host.com/lobcder-worker-1.0-SNAPSHOT rename the target: 

\begin{lstlisting}
mv target/lobcder-worker-1.0-SNAPSHOT target/lobcder-worker
\end{lstlisting}

Copy the target/lobcder-worker folder on \$CATALINA_HOME/webapps/ and restart tomcat.

\end{document}